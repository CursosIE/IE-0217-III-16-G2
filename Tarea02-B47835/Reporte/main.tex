\documentclass[11pt]{article}

\usepackage[letterpaper, margin=1in]{geometry}

\usepackage[spanish]{babel}
\usepackage[utf8]{inputenc}
\usepackage{multirow}
\usepackage{tabularx}
\usepackage{longtable}



%Figuras
\usepackage{graphicx, subfigure}
\usepackage[]{tikz}
\usepackage{pbox}

%Matemática
\usepackage{amsmath}
\usepackage{amssymb}

%Símbolos mate extra (alfabetos, etc.)
\usepackage{mathrsfs}


%Algoritmos
\usepackage{float}
\usepackage{algorithm}
\usepackage{algorithmicx}
\usepackage{algpseudocode}
\usepackage{listings}
\usepackage{xcolor}
\lstset{language=C++,
basicstyle=\footnotesize,
                keywordstyle=\color{blue}\ttfamily,
                stringstyle=\color{red}\ttfamily,
                commentstyle=\color{green}\ttfamily,
                morecomment=[l][\color{magenta}]{\#}
}


\usepackage{color}
\usepackage{hyperref}

\usepackage{mdframed}
\usepackage{tcolorbox}
\usepackage{multicol}
\usepackage{booktabs}
\usepackage{tabulary}
\definecolor{darkblue}{rgb}{0 , 0.054 , 0.196}



\title{Reporte Tarea 2}
\author{Ericka Z\'u\~niga Calvo - B47835}
\date{23 de enero de 2017}

\begin{document}

\maketitle
\hrule
\hrule
\tableofcontents
\hspace{5mm}
\hrule
\hrule


\section{Enunciados}

\begin{enumerate}
\item Implemente una clase en C++ que modele polinomios, utilizando sobrecarga de operadores que efectúe las cuatro operaciones básicas algebraicas:

\begin{itemize}
\item P\& operator+(const P\& other);
\item P\& operator-(const P\& other);
\item P\& operator*(const P\& other);
\item P\& operator/(const P\& other);\\
Para la divisi\'on utilice el m\'etodo de la divisi\'on sint\'etiica con sus restricciones.
\end{itemize}

\item Haga un programa de prueba en donde se creen dos objetos e interactuen entre ellos.
\item Escriba tambi\'en un Makefile con al menos tres reglas:
\begin{itemize}
\item build: para compilar el programa.
\item run: para ejecutar el programa.
\item clean: para eliminar ejecutables y archivos intermedios.
\end{itemize}
\end{enumerate}

\section{C\'odigo}
\begin{enumerate}
\item Polinomio.h
\begin{lstlisting}
#ifndef POLINOMIO_H
#define POLINOMIO_H
#include <iostream>
using namespace std;

class Polinomio {
public:

    template<size_t X> Polinomio(int (&arr)[X])
    {
        grado = X-1;
        data = new int[grado+1];
        for (int i=0;i<X;i++)
        {
            data[i]=arr[i];
        }
    }
    Polinomio(int arr[], int sizeArr);
    Polinomio(const Polinomio& orig);
    virtual ~Polinomio();
    void result();

    Polinomio& operator+(const Polinomio &other);
    Polinomio& operator-(const Polinomio &other);
    Polinomio& operator*(const Polinomio &other);
    Polinomio& operator/(const Polinomio &other);
    
    int grado;
    int* data;
    
};

#endif /* POLINOMIO_H */

\end{lstlisting}

\item Polinomio.cpp
\begin{itemize}
\item Se incluye el header de la clase polinomio.
\begin{lstlisting}
#include "Polinomio.h"
\end{lstlisting}

\item Polinomio(int arr[ ],int sizeArr)\\
Este es el constructor de la clase Polinomio, recibe como par\'ametros un arreglo y el tama\~no de ese arreglo.
Se inicializan los atributos del objeto polinomio.\\
\begin{lstlisting}
Polinomio::Polinomio(int arr[], int sizeArr) {
    grado = sizeArr-1;
    data = new int[grado+1];
    for (int i=0;i<sizeArr;i++)
    {
        data[i]=arr[i];
    }
}
\end{lstlisting}

\item Polinomio(const Polinomio \&orig)\\
Este es el constructor por copia de la clase Polinomio.\\
Recibe como par\'ametro un objeto de la clase Polinomio. 
\begin{lstlisting}
Polinomio::Polinomio(const Polinomio& orig) {
    grado = orig.grado;
    data = new int[grado+1];
    for(int i=0;i<orig.grado+1;i++){
        data[i]=orig.data[i];
    }
}
\end{lstlisting}

\item ~Polinomio()\\
Destructor de la clase polinomio. Elimina el arreglo data creado dinamicamente.
\begin{lstlisting}
Polinomio::~Polinomio() {
    delete[] data;
}
\end{lstlisting}

\item result()\\
Se encarga de imprimir el polinomio resultante.\\
\begin{lstlisting}
void Polinomio::result(){
    for(int i=grado;i>0;i--){
        if (data[i-1] >= 0) 
        {
            cout<<data[i]<<"x^"<<i<<"+";
        }
        else 
        {
            cout<<data[i]<<"x^"<<i;
        }
    }
    cout<<data[0]<<endl;
}
\end{lstlisting}

\item operator +:\\
Esta es la sobrecarga del operador +, para poder ser utilizado con objetos de tipo polinomio.\\
Inicialmente se compara si el grado de ambos polinomios es igual. Si lo es, simplemente se suman los elementos de la misma posicion en data de cada uno de los polinomios. Y esto es guardado en un arreglo temporal. Dicho arreglo sirve como par\'ametro para la creaci\'on del polinomio resultante.

Si el grado es diferente, se crea un puntero al polinomio de mayor grado, y se crea una variable menor grado perteneciente al polinomio restante. La suma se realiza como fue especificado anteriormente, hasta que alcance el valor de menor grado. Luego desde menor grado hasta el grado de mayor, se va guardando en la posicion correspondiente el valor de data en esa posicion del puntero mayor.\\
\\
\begin{lstlisting}
Polinomio& Polinomio::operator+(const Polinomio &other)
{
    if (grado == other.grado)
    {
        int* tempArray = new int[grado+1];
        
        for(int i=0; i<grado+1;i++)
        {
            tempArray[i]= data[i]+other.data[i];
        }
        
        Polinomio* result = new Polinomio(tempArray,grado+1);
        
        delete[] tempArray;
        return *result;
    }
    else
    {
        int menorGrado;
        const Polinomio* mayor;
        if (other.grado > grado)
        {
            menorGrado = grado;
            mayor = &other;
        }
        else
        {
            menorGrado = other.grado;
            mayor = this;
        }
        int* tempArray = new int[mayor->grado+1];
            for(int i=0; i<menorGrado+1;i++)
            {
                tempArray[i]= data[i]+other.data[i];
            }
            for(int i=menorGrado+1;i<mayor->grado+1;i++)
            {
                tempArray[i]=mayor->data[i]; 
            }
          
            Polinomio* result = new Polinomio(tempArray,mayor->grado+1);
        
            delete[] tempArray;
            return *result;
    }
}
\end{lstlisting}

\item operator -:\\
El operador - tiene una gran similitud con el operador +. Pero tiene una condici\'on, si grado mayor es igual a other, entonces en el caso de que los grados de los polinomios sean diferentes, se multiplica por -1, los valores que se encuentran en las posiciones entre menor grado y el grado de mayor.
\\
\\
\begin{lstlisting}

Polinomio& Polinomio::operator-(const Polinomio &other)
{
    if (grado == other.grado)
    {
        int* tempArray = new int[grado+1];
        
        for(int i=0; i<grado+1;i++)
        {
            tempArray[i]= data[i]-other.data[i];
        }
        
        Polinomio* result = new Polinomio(tempArray,grado+1);
        
        delete[] tempArray;
        return *result;
    }
    else
    {
        int menorGrado;
        const Polinomio* mayor;
        if (other.grado > grado)
        {
            menorGrado = grado;
            mayor = &other;
        }
        else
        {
            menorGrado = other.grado;
            mayor = this;
        }
        int* tempArray = new int[mayor->grado+1];
            for(int i=0; i<menorGrado+1;i++)
            {
                tempArray[i]= data[i]-other.data[i];
            }
            for(int i=menorGrado+1;i<mayor->grado+1;i++)
            {
                if(mayor->grado==grado)
                {
                    tempArray[i]=mayor->data[i];
                }
                else
                {
                    tempArray[i]=mayor->data[i]*(-1);
                } 
            }
          
            Polinomio* result = new Polinomio(tempArray,mayor->grado+1);
        
            delete[] tempArray;
            return *result;
    }
}
\end{lstlisting}

\item operator *:\\
Este es el operador * de la funci\'on polinomio. Para su soluci\'on, inicialmente se crea un arreglo dinamico temporal de tama\~no de la suma de los grados de los dos polinomios +1.Posteriormente se van multiplicando cada uno de los elementos del data del polinomio1, con cada uno de los elementos del data del polinomio 2. Y estos se van guardando (+=) en la posicion en el arreglo temporal que corresponde a la suma de cada una de las posiciones en el data de sus polinomios.
Se crea un objeto de tipo polinomio, que utiliza como atributo el arreglo y su tama\~no.
\begin{lstlisting}
Polinomio& Polinomio::operator*(const Polinomio &other)
{   
    int nuevoGrado = grado+other.grado;
    int* tempArray = new int[nuevoGrado+1] ();
        
        for(int i=0; i<grado+1;i++)
        {
            for (int j=0;j<other.grado+1;j++)
            {
                int temp = i+j;
                tempArray[temp]+= data[i]*other.data[j];
            }
            
        }
        
        Polinomio* result = new Polinomio(tempArray,nuevoGrado+1);
        
        delete[] tempArray;
        return *result;
    
}

\end{lstlisting}

\item operator /:\\
Este es el operador / de la clase polinomio. Este es construido con el m\'etodo de divisi\'on sint\'etica, el cual cuenta con las restricciones: El dividendo debe tener grado mayor que uno, El divisor debe tener grado igual a 1, y coeficiente en dicha posici\'on de 1.

Al confirmar dichas restricciones, se inicia con el algoritmo. Se crea un arreglo temporal con tama\~no de la resta entre el grado del dividendo y el grado del divisor. Y se toma el coeficiente en la posicion m\'as alta del dividendo, como una variable temporal y esta se guarda en la posici\'on m\'as alta del arreglo temporal. Se inicia un ciclo, que inicia en el grado del dividendo -1 y este va decreciendo hasta alcanzar el valor de 0.Luego la variable temporal se multiplica por el valor en la posici\'on m\'as peque\~na del divisor, y se guarda en temp2. Y se realiza la suma entre el elemento en data en la posicion [i] y el valor temp2, esto se guarda en la posicion en el arreglo de i-1 y en la variable temporal.\\
Se retorna un objeto Polinomio que recibe como par\'ametro el arreglo temporal.
\begin{lstlisting}
 Polinomio& Polinomio::operator/(const Polinomio &other){
     Polinomio* result = NULL;
     if(other.grado!=1||grado<1||other.data[1]!=1)
     {
         cout<<"No se puede realizar dicha division por el metodo de division sintetica."<<endl;
         result = new Polinomio(NULL,0);
     }
     else{
         int* tempArray = new int[grado-other.grado+1];
         int temp = data[grado];
         int div = other.data[0]*(-1);
         tempArray[grado-other.grado]=temp;
         for(int i=grado-1;i>0;i--){
             int temp2= temp*div;
             tempArray[i-1]=data[i]+temp2;
             temp=tempArray[i-1];
             
         }
         result = new Polinomio(tempArray,grado-other.grado+1);
         delete[] tempArray;
     }
     return *result;
 }
\end{lstlisting}
\end{itemize}

\item Main
\\Programa de prueba, donde se crean dos objetos de la clase polinomio. Y se suman, restan, multiplican y dividen entre si. 
As\'i como se logra imprimir su resultado.

\begin{lstlisting}
#include <cstdlib>
#include "Polinomio.h"
using namespace std;

int main(int argc, char** argv) {

    int array1[] = {-16,2,0,-8,1};
    int array2[] = {-8,1};
    
    Polinomio* p1=new Polinomio(array1);
    Polinomio* p2=new Polinomio(array2);
    
    Polinomio p3 = *p2+*p1;
    p3.result();

    Polinomio p4 = *p2-*p1;
    p4.result();

    Polinomio p5 = (*p1)*(*p2);
    p5.result();
    
    Polinomio p6 = (*p1)/(*p2);
    p6.result();
    delete p1;
    delete p2;
    return 0;
}
\end{lstlisting}
\end{enumerate}

\section{Conclusiones}
\begin{itemize}
\item Se logr\'o sobrecargar los operadores suma, resta, multiplicaci\'on y divisi\'on con \'exito para la clase Polinomio.
\end{itemize}

\end{document}
