\documentclass[11pt]{article}

\usepackage[letterpaper, margin=1in]{geometry}

\usepackage[spanish]{babel}
\usepackage[utf8]{inputenc}
\usepackage{multirow}
\usepackage{tabularx}
\usepackage{longtable}



%Figuras
\usepackage{graphicx, subfigure}
\usepackage[]{tikz}
\usepackage{pbox}

%Matemática
\usepackage{amsmath}
\usepackage{amssymb}

%Símbolos mate extra (alfabetos, etc.)
\usepackage{mathrsfs}


%Algoritmos
\usepackage{float}
\usepackage{algorithm}
\usepackage{algorithmicx}
\usepackage{algpseudocode}
\usepackage{listings}
\usepackage{xcolor}
\lstset{language=C++,
basicstyle=\footnotesize,
                keywordstyle=\color{blue}\ttfamily,
                stringstyle=\color{red}\ttfamily,
                commentstyle=\color{green}\ttfamily,
                morecomment=[l][\color{magenta}]{\#}
}


\usepackage{color}
\usepackage{hyperref}

\usepackage{mdframed}
\usepackage{tcolorbox}
\usepackage{multicol}
\usepackage{booktabs}
\usepackage{tabulary}
\definecolor{darkblue}{rgb}{0 , 0.054 , 0.196}



\title{Reporte Tarea 1}
\author{Ericka Z\'u\~niga Calvo - B47835}
\date{20 de enero de 2017}

\begin{document}

\maketitle
\hrule
\hrule
\tableofcontents
\hspace{5mm}
\hrule
\hrule


\section{Enunciados}
Implemente un programa en C++ utilizando herencia, que modele las tres aves legendarias de Pok\'emon: Articuno, Moltres, Zapdos.

\begin{enumerate}
\item Implemente una clase abstracta Pokemon que contenga:
\begin{itemize}
\item Atributos: name, species,HP,ATK,DEF,sATK,sDEF,SPD,EXP,call.
\item M\'etodos:
\begin{itemize}
\item virtual void atk1(Pokemon \&other)=0
\item virtual void atk2(Pokemon \&other)=0
\item virtual void atk3(Pokemon \&other)=0
\item virtual void atk4(Pokemon \&other)=0
\item string call
\item void Info()
\end{itemize}
\end{itemize}
\item Implemente cuatro clases, llamadas: Electric, Flying, Fire, Water que modelen los tipos de las aves m\'iticas. Estas clases herendan de la clase Pokemon.
\begin{itemize}
\item M\'etodos:
\begin{itemize}
\item static string getType()
\item static string getStrongVs()
\item static string getWeakVs()
\end{itemize}
\end{itemize}
\item Implemente tres clases concretas llamadas: Zapdos, Moltres, Articuno.
\begin{itemize}
\item Atributos: Type, strongVs,weakVs.
\item M\'etodos:
\begin{itemize}
\item void print
\item void atk1(Pokemon \&other);
\item void atk2(Pokemon \&other);
\item void atk3(Pokemon \&other);
\item void atk4(Pokemon \&other);

\end{itemize}
\end{itemize}
\item Haga un programa de prueba en donde se creen dos objetos e interactuen entre ellos.
\item Escriba tambi\'en un Makefile con al menos tres reglas:
\begin{itemize}
\item build: para compilar el programa.
\item run: para ejecutar el programa.
\item clean: para eliminar ejecutables y archivos intermedios.
\end{itemize}
\end{enumerate}

\section{C\'odigo}

A continuaci\'on se explicar\'an las clases Pokemon, Electric y Zapdos, junto con sus correspondientes definiciones, m\'etodos y atributos. Se explican \'unicamente estas, ya que son representativas de las dem\'as clases correspondientes.

\subsection{ Clase Pokemon:}
\begin{itemize}
\item Pokemon.h:
\begin{lstlisting}
#ifndef POKEMON_H
#define POKEMON_H
#include<iostream>
#include <vector>

class Pokemon {
public:
    Pokemon();
    virtual ~Pokemon();
    std::string name;
    std::string species;
    int HP;
    int ATK;
    int DEF;
    int sATK;
    int sDEF;
    int SPD;
    int EXP;
    std::string cry;

    void inf();
    std::string doCry();
    virtual void atk1(Pokemon &other)=0;
    virtual void atk2(Pokemon &other)=0;
    virtual void atk3(Pokemon &other)=0;
    virtual void atk4(Pokemon &other)=0;
    virtual std::vector<std::string> getTypes()=0;

    double getModifier(double stab, double type);
    int getDamage(int pwr,int atk, int def,double stab, double type);
    static double fRand(double fMin, double fMax);
    bool isCritHit(int speed);
    bool hasMissed(int accuracy);

};

#endif /* POKEMON_H */

\end{lstlisting}
\item Pokemon.cpp
\begin{lstlisting}
#include "Pokemon.h"
#include <iostream>
#include <stdlib.h>
#include <time.h>

Pokemon::Pokemon() {
    // std::cout << "Este es el constructor de mi Pokemon" << std::endl;
}

Pokemon::~Pokemon() {
    // std::cout << "Este es el destructor de mi Pokemon" << std::endl;
}

void Pokemon::inf() {
    std::cout << "Name:  " << name << std::endl;
    std::cout << "Species: " << species << std::endl;
    std::cout << "HP:  " << HP << std::endl;
    std::cout << "ATK:  " << ATK << std::endl;
    std::cout << "DEF: " << DEF << std::endl;
    std::cout << "sATK: " << sATK << std::endl;
    std::cout << "sDEF: " << sDEF << std::endl;
    std::cout << "SPD: " << SPD << std::endl;
    std::cout << "EXP: " << EXP << std::endl;
    std::cout << "cry" << cry << std::endl;
};

double Pokemon::getModifier(double stab, double type) {
    srand(time(NULL));
    int critMult = 1;
    if (isCritHit(SPD)) {
        critMult = 2;
    }
    return stab * type * critMult * fRand(0.85, 1.0);
}

int Pokemon::getDamage(int pwr, int atk, int def, double stab, double type) {
    double modifier = getModifier(stab, type);
    double damage = ((12 / 250)*(atk / def) * pwr + 2) * modifier * 5;
    return (int) damage;

}

double Pokemon::fRand(double fMin, double fMax) {
    double f = (double) rand() / (double) (RAND_MAX);
    return fMin + f * (fMax - fMin);
}

bool Pokemon::isCritHit(int speed) {
    double treshold = speed / 2;
    srand(time(NULL));
    double x = fRand(0.0, 255.0);
    if (x <= treshold) {
        return true;
        std::cout << "A critical hit!" << std::endl;
    } else {
        return false;
    }
}

bool Pokemon::hasMissed(int accuracy) {
    srand(time(NULL));
    double x = fRand(0.0, 1.0);
    std::cout << "El valor de x es" << x << std::endl;
    if (x > accuracy / 100) {
        return true;
    } else {
        return false;
    }
}

std::string Pokemon::doCry() {
    return cry;
}
\end{lstlisting}

\begin{itemize}
\item Bibliotecas utilizadas:
\begin{itemize}
\item iostream: Se encarga del flujo de entrada y salida de datos.
\item vector: Se utiliza para operar con arreglos unidimensionales de datos.
\item stdlib.h: En este caso se utiliza para utilizar la funci\'on de C++ rand(), es decir, generar un n\'umero aleatorio.
\item time.h: Es para plantear la semilla,la cual en este caso es el reloj de la m\'aquina, que origina los n\'umeros pseudoaleatorios. 
\end{itemize}
\item M\'etodos:
\begin{itemize}
\item Pokemon():
Constructor de la clase Pokemon.
\item $\sim$ Pokemon():
Destructor de la clase Pokemon.
\item inf():
Se encarga de imprimir todos los atributos correspondientes a los Objetos de la clase Pokemon.
\item doCry():
Retorna como string el canto del objeto pokemon.
\item virtual atk\#(Pokemon \&other)=0:
Genera m\'etodos virtuales puros (atk1,atk2,atk3,atk4), los cuales posteriormente ser\'an inicializados en las clases Zapdos, Moltres y Articuno. De manera que cada uno puede definirlo a su conveniencia y recibe como par\'ametro un objeto de tipo Pokemon.
\item virtual vector string getTypes()=0:
Al igual que el m\'etodo anterior, es virtual puro. Y se creo para que retorne un vector de tipo string que contiene los tipos de las especies(Zapdos,Moltres,...).
\item getModifier, getDamage:
Son m\'etodos que retornan el resultado de una funci\'on matem\'atica, de manera que se obtiene el da\~no que hace un ataque al Pokemon oponente. 
\item fRand: 
Genera un n\'umero aleatorio entre fMin y fMax.
\item isCritHit:
En base a la generaci\'on de un n\'umero aleatorio y su comparaci\'on con un n\'umero fijo (treshold) dice si el golpe es cr\'itico o no.
\item hasMissed:
Toma como par\'amentro la precisi\'on del golpe, y en base a la generaci\'on de un n\'umero aleatorio, retorna si el ataque fue exitoso o no.
\item cry
Retorna una string que contiene el canto del Pokemon.
\end{itemize}
\end{itemize}
\end{itemize}
\subsection{Clase Electric:} 
Esta clase hereda de la clase Pokemon.
\begin{itemize}
\item Electric.h
\begin{lstlisting}

#ifndef ELECTRIC_H
#define ELECTRIC_H
#include <iostream>
#include "Pokemon.h"

class Electric : virtual public Pokemon{
public:
  Electric();
  virtual ~Electric();
  static std::string getType();
  static std::string getStrongVs();
  static std::string getWeakVs();
} ;

#endif /* ELECTRIC_H */
\end{lstlisting}

\item Electric.cpp
\begin{lstlisting}
#include "Electric.h"

Electric::Electric() {
    // std::cout<<"Constructor de pokemon tipo electrico"<<std::endl;
}

Electric::~Electric() {
    // std::cout<<"Destructor de pokemon tipo electrico"<<std::endl;
}

std::string Electric::getType() {
    return "Electric";
}

std::string Electric::getStrongVs() {
    return "Water and Flying";
}

std::string Electric::getWeakVs() {
    return " ";
}

\end{lstlisting}
\item Bibliotecas utilizadas:
\begin{itemize}
\item iostream.
\item Pokemon.h: Este header es necesario, para hacer que la clase Electric herede de Pokemon.
\end{itemize}
\item M\'etodos:
\begin{itemize}
\item Electric(): 
Constructor.
\item $\sim$Electric():
Destructor.
\item getType():
Retorna una string que contiene el tipo de Pokemon.

\item getStrongVs():
Retorna una string que contiene los tipos de Pokemon a los que es fuerte al enfrentarse dicho tipo, en este caso Electric.

\item getWeakVs():
Retorna una string que contiene los tipos de Pokemon a los que es debil al enfrentarse dicho tipo, en este caso Electric.


\end{itemize}
\end{itemize}
\subsection{ Clase Zapdos.}
Esta clase hereda tanto de la clase Electric, como de la clase Flying.
\begin{itemize}
\item Zapdos.h
\begin{lstlisting}
#ifndef ZAPDOS_H
#define ZAPDOS_H

#include <iostream>
#include <vector>
#include "Pokemon.h"
#include "Electric.h"
#include "Flying.h"

class Zapdos : public Electric, public Flying{
public:
  Zapdos();
  virtual ~Zapdos();
  std::string type;
  std::string strongVs;
  std::string weakVs;
  void atk1(Pokemon &other);
  void atk2(Pokemon &other);
  void atk3(Pokemon &other);
  void atk4(Pokemon &other);
  std::vector<std::string> getTypes();
  void printInf();
  static void gen_random(char *s, const int len);
};

#endif /* ZAPDOS_H */
\end{lstlisting}
\item Zapdos.cpp
\begin{lstlisting}
#include "Zapdos.h"

static const char alphanum[] =
        "0123456789"
        "ABCDEFGHIJKLMNOPQRSTUVWXYZ";

int stringLength = sizeof (alphanum) - 1;

Zapdos::Zapdos() {
    //std::cout<<"Constructor Zapdos"<<std::endl;
    type = Electric::getType() + "/" + Flying::getType();
    strongVs = Electric::getStrongVs() + Flying::getStrongVs();
    weakVs = Electric::getWeakVs() + Flying::getWeakVs();
    char* tempName = new char;
    gen_random(tempName, 4);
    name = tempName;
    delete tempName;
    species = "Zapdos";
    cry = "TUNUNUNUN";
    HP = 90;
    ATK = 90;
    DEF = 85;
    sATK = 125;
    sDEF = 90;
    SPD = 100;
    EXP = 10;
}

Zapdos::~Zapdos() {
    // std::cout<<"Destructor Zapdos"<<std::endl;
}

void Zapdos::gen_random(char *s, const int len) {
    static const char alphanum[] =
            "0123456789"
            "ABCDEFGHIJKLMNOPQRSTUVWXYZ";

    for (int i = 0; i < len; ++i) {
        s[i] = alphanum[rand() % (sizeof (alphanum) - 1)];
    }

    s[len] = 0;
}

std::vector<std::string> Zapdos::getTypes() {
    std::vector<std::string> types;
    types.push_back(Electric::getType());
    types.push_back(Flying::getType());
    return types;
}

void Zapdos::atk1(Pokemon &other) {
    std::cout << name << ":" << species << " attack " << other.name 
    << ":" << other.species << std::endl;
    
    std::cout << name << ":" << species << " used DRILL PECK!" << std::endl;
    int pwr = 80;
    int acc = 100;
    std::vector<std::string> oType = other.getTypes();
    double mult = 1;
    bool isElectric = false;
    int i = 0;
    while (!isElectric && i < 2) {
        if (oType[i].compare("Electric") == 0) {
            std::cout << "It's not very effective..." << std::endl;
            mult = 0.5;
            isElectric = true;
        }
        i++;
    }
    int damage = getDamage(pwr, ATK, DEF, 1.5, mult);

    other.HP = other.HP - damage;
    
    std::cout << "The damage caused by " << name << ":" << species << " to" 
    << other.name << ":" << other.species << " is: " << damage << std::endl;
    
    std::cout << "------------------------------------------" << std::endl;
    std::cout << "" << std::endl;
    std::cout << "" << std::endl;
}

void Zapdos::atk2(Pokemon &other) {
    std::cout << name << ":" << species << " attack " << other.name 
    << ":" << other.species << std::endl;
    
    std::cout << name << ":" << species << " used THUNDER SHOCK!" << std::endl;
    int pwr = 40;
    int acc = 100;
    std::vector<std::string> oType = other.getTypes();
    double mult = 2;
    bool isWater = false;
    int i = 0;
    while (!isWater && i < 2) {
        if (oType[i].compare("Water") == 0) {
            mult = 4;
            isWater = true;
        }
        i++;
    }
    std::cout << "It's super effective!" << std::endl;

    int damage = getDamage(pwr, sATK, sDEF, 1.5, mult);
    other.HP = other.HP - damage;
    std::cout << "The damage caused by " << name << ":" << species << " to" 
    << other.name << ":" << other.species << " is: " << damage << std::endl;
    
    std::cout << "------------------------------------------" << std::endl;
    std::cout << "" << std::endl;
    std::cout << "" << std::endl;
}

void Zapdos::atk3(Pokemon &other) {
    std::cout << name << ":" << species << " attack " << other.name <<
    ":" << other.species << std::endl;
    
    std::cout << name << ":" << species << " used ZAP CANNON!" << std::endl;
    int pwr = 120;
    int acc = 50;
    if (hasMissed(acc)) {
        std::cout << name << ":" << species << "'s attack missed!" << std::endl;
        std::cout << "------------------------------------------" << std::endl;
        std::cout << "" << std::endl;
        std::cout << "" << std::endl;
    } else {
        std::vector<std::string> oType = other.getTypes();
        double mult = 2;
        bool isWater = false;
        int i = 0;
        while (!isWater && i < 2) {
            if (oType[i].compare("Water") == 0) {
                mult = 4;
                isWater = true;
            }
            i++;
        }
        std::cout << "It's super effective!" << std::endl;
        int damage = getDamage(pwr, sATK, sDEF, 1.5, mult);
        other.HP -= damage;
        std::cout << "The damage caused by " << name << ":" << species << " to" <<
        other.name << ":" << other.species << " is: " << damage << std::endl;
        
        std::cout << "------------------------------------------" << std::endl;
        std::cout << "" << std::endl;
        std::cout << "" << std::endl;
    }

}

void Zapdos::atk4(Pokemon &other) {
    std::cout << name << ":" << species << " attack " << other.name <<
    ":" << other.species << std::endl;
    
    std::cout << name << ":" << species << " used THUNDER!" << std::endl;
    int pwr = 110;
    int acc = 70;
    if (hasMissed(acc)) {
        std::cout << name << ":" << species << "'s attack missed!" << std::endl;
        std::cout << "------------------------------------------" << std::endl;
        std::cout << "" << std::endl;
        std::cout << "" << std::endl;
    } else {
        std::vector<std::string> oType = other.getTypes();
        double mult = 2;
        bool isWater = false;
        int i = 0;
        while (!isWater && i < 2) {
            if (oType[i].compare("Water") == 0) {
                mult = 4;
                isWater = true;
            }
            i++;
        }
        std::cout << "It's super effective!" << std::endl;
        int damage = getDamage(pwr, sATK, sDEF, 1.5, mult);
        other.HP -= damage;
        std::cout << "The damage caused by " << name << ":" << species << " to" <<
        other.name << ":" << other.species << " is: " << damage << std::endl;
        
        std::cout << "------------------------------------------" << std::endl;
        std::cout << "" << std::endl;
        std::cout << "" << std::endl;
    }
}

void Zapdos::printInf() {
    std::cout << "Name:  " << name << std::endl;
    std::cout << "Species: " << species << std::endl;
    std::cout << "HP:  " << HP << std::endl;
    std::cout << "ATK:  " << ATK << std::endl;
    std::cout << "DEF: " << DEF << std::endl;
    std::cout << "sATK: " << sATK << std::endl;
    std::cout << "sDEF: " << sDEF << std::endl;
    std::cout << "SPD: " << SPD << std::endl;
    std::cout << "EXP: " << EXP << std::endl;
    std::cout << "cry: " << cry << std::endl;
    std::cout << "------------------------------------------" << std::endl;
    std::cout << "" << std::endl;
    std::cout << "" << std::endl;
}
\end{lstlisting}
\item Bibliotecas utilizadas:
\begin{itemize}
\item iostream
\item vector
\item Pokemon.h
\item Electric.h
\item Flying.h

\end{itemize}
\item M\'etodos:
\begin{itemize}
\item Zapdos()
\item $\sim$Zapdos()
\item atk1 : 
Se inicializa el m\'etodo atk1, el cual recibe como parametro un objeto de la clase Pokemon.
\item getTypes(): Se crea un vector que contiene los tipos de Pokemon de Zapdos.
\item gen\_random: Genera un string de tama\~no len, la cual contiene caracteres de tipo num\'ericos o alfabeticos.
\item printInf(): Imprime la informaci\'on relevante del Pokemon.
\end{itemize}
\end{itemize}

\section{Conclusiones}
\begin{itemize}
\item Se logr\'o utilizar herencia para crear distintas clases que utilicen los m\'etodos y atributos de sus clases padres.
\item Se implementaron distintas clases con herencia m\'ultiple.
\item Se utilizaron m\'etodos virtuales, declarados como virtuales puros en la clase padre, para ser utilizados por cada uno de sus clases hijos.
\end{itemize}






\newpage
\bibliographystyle{plain}
\bibliography{bibliography}



\end{document}
